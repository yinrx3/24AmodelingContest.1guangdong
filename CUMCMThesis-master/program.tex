\documentclass[10pt, a4paper, oneside]{ctexart}
\usepackage{amsmath, amsthm, amssymb, appendix, bm,fancyhdr}
\usepackage{diagbox, graphicx,geometry, hyperref, lipsum,longtable, mathrsfs,titlesec,textcomp, zhnumber}

\renewcommand{\abstractname}{\Large\textbf{摘要}}
\renewcommand{\thesection}{\zhnum{section}}
\renewcommand{\thesubsection}{\arabic{section}.\arabic{subsection}}
\renewcommand{\theequation}{\arabic{section}.\arabic{equation}}
\renewcommand{\thetable}{\arabic{section}.\arabic{table}}
\renewcommand{\thefigure}{\arabic{section}.\arabic{figure}}
\newtheorem{theorem}{定理 \arabic{section}.\hspace{-5pt}}
\newtheorem{definition}{定义 \arabic{section}.\hspace{-5pt}}
\newtheorem{lemma}{引理 \arabic{section}.\hspace{-5pt}}
\newtheorem{corollary}{推论 \arabic{section}.\hspace{-5pt}}
\newtheorem{example}{例 \arabic{section}.\hspace{-5pt}}
\newtheorem{proposition}{命题 \arabic{section}.\hspace{-5pt}}
\title{\textbf{这里是题目}}
\author{张洋, 殷润轩, 邓俊辉}
\date{\today}
\linespread{1.5}
\geometry{left=2.9cm, right=2.9cm, top=3.18cm, bottom=3.18cm}
\begin{document}
\maketitle
\setcounter{page}{0}
\thispagestyle{empty}
\begin{abstract}
	{\fontsize{10}{15}\selectfont\
        针对问题1,运用几何分析与螺线方程,可以求解龙身前把手与后把手实时位置与时间之间的关系。对于板凳的任意两点,满足以下式子,进而能够求解出问题1.\par
        针对问题2,考虑板凳相互碰撞的约束条件,将其分为速度大小相反的两种模型。根据实际情况,我们选择了模型2作为参考的模型,并且用MATLAB对其进行模拟,代码运行得出具体的数据,将  带入   中,借助问题1所给出的式子,计算出最终的结果。
	\textbf{关键词}:
	\space 螺线 、微分方程 、速度\newpage}
\end{abstract}
\pagestyle{fancy}
\fancyhf{}
\fancyheadoffset{0cm}
\fancyfoot[C]{\thepage}
\newgeometry{left=2.5cm, right=2.5cm, top=3cm, bottom=3cm}
\section{问题背景与重述}
本文所研究的问题基于“板凳龙”这一独特的文化活动。“板凳龙”又称“盘龙”,这是一种在浙闽地区流传已久的传统地方民俗文化活动。当地的居民将几十条乃至上百条的板凳首尾相连,形成蜿蜒曲折的“盘龙”,整体呈圆盘状。\par
本文所研究的问题正是基于这一文化活动给出的具体化数学问题。问题要求我们对一条整体为螺线结构的“盘龙”进行分析,从实际的物理现象中抽离出数学物理模型,建立方程并求解。题目给出了每条板凳的结构与相关数据。\par
问题1给出了舞龙队行进时的速度、螺线的螺距、龙头的初始位置等多个条件,并要求我们解出0时刻、1分钟、2分钟、3分钟、4分钟以及5分钟时刻多个龙身前把手与后把手的速度与坐标。
问题2要求我们给出舞龙队行进的终止时间,使得板凳之间不发生碰撞,并给出多个龙身前把手与后把手的速度与坐标。

\end{document}